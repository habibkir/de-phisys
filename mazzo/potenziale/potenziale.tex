\documentclass{article}

\usepackage{amsmath}
\usepackage{amsfonts}
\usepackage{amssymb}

\begin{document}

\subsection{nota sulle notazioni}
Si ricordi un attimo che in analisi 1
\[\int 1 dx = x\]
la cosa può essere scritta come
\[\int dx = x\]
o, se sei un fisico, come
\[ \text{ diciamo che se ho } \int qualcosa = x \text{ allora chiamiamo quel qualcosa } dx\]
è spesso facile dimenticarsi di questo fatto, visto che era nel programma spiegato dal Cinti, ma va tenuto a mente visto che la notazione in questione è abusata all'inteno di questo corso.

Sperando di ricordarsi da fisica 1 che
\[W=-\Delta U\]
Per ricapire un po' meglio sta cosa, molti conconrderanno sul fatto che il lavoro per portare un frigo al dal piano terra al terzo piano può essere considerato positivo, e il frigo ha più potenziale al terzo piano che al piano terra (può scassare più roba se cade), il lavoro per portarlo dal terzo piano al piano terra sarà invece negativo, stai lavorando con la gravità (teoricamente), e il frigo ha meno potenziale al piano terra che al terzo piano, quindi
\[W_{piano\ terra,\ terzo\ piano} > 0 \iff (U_{piano\ terra}-U_{terzo\ piano})<0\]
\[W_{terzo\ piano,\ piano\ terra} < 0 \iff (U_{terzo\ piano}-U_{piano\ terra})>0\]

\end{document}
