\documentclass{article}

\usepackage{amsmath}
\usepackage{amsfonts}
\usepackage[utf8]{inputenc}

\begin{document}
(Vi sono alcune parentesi extra nelle formule sono puramente per leggibilità, perdonate possibili malintesi)

La forza di Coulomb è una copia spiccicata della gravitazione di Newton, ma per le cariche
Se ho una particella di carica $q_1$ e una di carica $q_2$ la forza applicata su $q_2$ da $q_1$ sarà

\begin{equation*}
  \overrightarrow{F_{2,1}} = k_0 \frac{q_1 q_2}{(r_{2,1})^2}\overrightarrow{r_{2,1}}
\end{equation*}

con $q_1$ e $q_2$ pari alle cariche tra le quali si ha forza, $r$ pari alla distanza tra le due cariche, e $\overrightarrow{r}$ versore tra le due cariche 
possiamo vedere dalla formula (e dalla seconda legge di Newton) che il di $\overrightarrow{F_{1,2}}$ sarà pari a quello di $\overrightarrow{F_{2,1}}$

Se ho una carica $q_1$ e voglio sapere che forza viene esercitata su di essa da altre due cariche $q_2$ e $q_3$, allora posso  possibile sommarle e ottenere
\begin{align*}
  \overrightarrow{F_{tot}} &= \overrightarrow{F_{2,1}} + \overrightarrow{F_{3,1}} \\
                          &= k_0 \frac{q_1 q_2}{(r_{2,1})^2}\overrightarrow{r_{2,1}} +
                             k_0 \frac{q_1 q_3}{(r_{3,1})^3}\overrightarrow{r_{3,1}} \\
                          &= k_0 q_1(\frac{q_2}{(r_{2,1})^2}\overrightarrow{r_{2,1}}+
                                     \frac{q_3}{(r_{3,1})^2}\overrightarrow{r_{3,1}})
\end{align*}

La forza esercitata su una $q_1$ può essere quindi espressa come `` $q_1 \times$ qualcosa che dipende solo da come sono messe le due cariche intorno''

questa formulazione si può estendere a $q_1$ con altre tre cariche intorno (chiamate con molta fantasia $q_2$ $q_3$ $q_4$), in cui la forza totale farà
%qui magari poi potresti mettere la parte in cui semplifichi le formule per la forza dalle singole cariche e fai vedere che bla bla bla t'ha capito
\begin{align*}
  \overrightarrow{F_{tot}} &= k_0 q_1 + \frac{q_2}{(r_{2,1})^2}\overrightarrow{r_{2,1}}
                                      + k_0 q_1 + \frac{q_3}{(r_{3,1})^2}\overrightarrow{r_{3,1}}
                                      + k_0 q_1 + \frac{q_4}{(r_{4,1})^2}\overrightarrow{r_{4,1}} \\
                           &= k_0 q_1 + (\frac{q_2}{(r_{2,1})^2}\overrightarrow{r_{2,1}}
                                      +  \frac{q_2}{(r_{3,1})^2}\overrightarrow{r_{3,1}}
                                      +  \frac{q_2}{(r_{4,1})^2}\overrightarrow{r_{4,1}})
\end{align*}

questa cosa si può estendere a una qualsiasi distribuzione di tot cariche $q_1$,$q_2$,...,$q_{tot}$ che agiscono su, facciamo, $q_0$, in questo caso la forza totale sarà

\begin{equation*}
  \overrightarrow{F_{tot}} = k_0 q_0 \sum_{i = 1}^{tot} \frac{q_i}{(r_{i,0})^2}
\end{equation*}

andando nel caso continuo fai tendere tutto a un infinito e/o un infinitesimo finchè non arrivi alla stessa cazzo di formula se non che c'è un integrale

\begin{equation*}
  \overrightarrow{F_{tot}} = k_0 q_0 \int_{\tau} \frac{q}{(r_{i,0})^2}
\end{equation*}

\end{document}
